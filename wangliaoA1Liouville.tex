\subsection{Notes by Wang: From Liouville to grand canonical ensembles}
\def\b#1{\mathbf{#1}}
\def\mref#1{\hspace{-7.5pt} ~(\ref{#1})}
\def\td#1{\frac{d#1}{dt}}
\footnote{The note is written by Wang and the first draft finished on $14^{th}$ Mar. 2020.}The Hamilton's equation for a system of $d$ degree of freedom may be written as
\begin{equation}\label{he}
\dot{\b x}=\b J^{-1}D H(\b x)
\end{equation}
where
\[
    \b x=\begin{pmatrix}\b q\\\b p\end{pmatrix}
    \]
is a vector of the space $\mathbb R^d\times\mathbb R^d=\mathbb R^{2d}$ and $\b q$, $\b p$ are generalized coordinates and generalized momentums.


\noindent $\b J$ is the matrix
\[
    \b J=\begin{pmatrix} 0 & -I\\ I & 0\end{pmatrix}
    \]
where $I$ is the identity on the space $\mathbb R^d$. $DH(\b x)$ is the differential of the Hamiltonian at the point $\b x$.

A small variation of the $\b x$ may be taken on the both side of the \mref{he}
\def\tdb#1{\td{(#1)}}
\def\vd{\Delta}
\[
\tdb{\b x+\Delta\b x}=\b J^{-1}D H(\b x+\Delta\b x)
\]
Let $\vd\b x$ be sufficiently small, then there seems to be
\def\dx{\vd\b x}
\[
\tdb{\b x+\dx}=\b J^{-1}D(H(\b x)+D(H(\b x))\dx)
\]
with \mref{he}
\[
\tdb{\dx}=\b J^{-1}D(D(H(\b x))\dx)
\]
The variation $\dx$ is independent of the $\b x$. Thus
\def\hex{\b H(\b x)}
\begin{equation}\label{vhe}
\tdb{\dx}=\b J^{-1}\hex\dx
\end{equation}
where $\hex$ is the Hessian matrix at point $x$
\[
    D(D(H(\b x)))=\hex=\left[\frac{\partial^2 H}{\partial x_i\partial x_j}\right]=\left[H_{,i,j}\right]
    \]
It is symmetric as long as the Hamiltonian is of class $C^{(2)}$.

If all the points in the space $\mathbb R^{2d}$ are considered as the inital conditions $\b x(0)$ for \mref{he}, then aftet time $t$ each point $\b x(0)$ corresponds to $\b x(t)$ provided that the solution to \mref{he} is unique. This permits the definition of the function $\phi_t$
\[
   \phi_t :  \mathbb R^{2d}\longrightarrow \mathbb R^{2d}, \text{\ \ \ and\ \ \ } \phi_t: \b x(0)\mapsto \b x(t) 
    \]

If the system is initially at point $\b x$ when $t=0$, then for sufficiently small $\b h$ varying at the same time $t=0$, there
\[
    \dx=\phi_t(\b x+\b h)-\phi_t(\b x)=D\phi_t(\b x)\b h
    \]
 with \mref{vhe}, there
 \begin{equation}\label{vhe2}
 \tdb{D\phi_t(\b x)}\b h=\b J^{-1}\hex D\phi_t(\b x)\b h
 \end{equation}
 with $\b h$ varying in an arbitrary manner
 \begin{equation}\label{vhe2}
 \tdb{D\phi_t(\b x)}=\b J^{-1}\hex (D\phi_t(\b x))
 \end{equation}
 At point $\b x$, it may be set up a local reference frame with bases from the family
 \[
     \mathfrak B=\{\b h_1, \b h_2,\cdots ,\b h_{2d}\}
     \]
with members sufficiently small such that the linearity is a prominent feature.

Denote by $\Omega(0,\b x)$ the oriented volume determined by $\mathfrak B$, which is
\[
    \det(\b h_1, \b h_2,\cdots ,\b h_{2d})=\Omega(0,\b x)
    \] 
After time $t$, the coordinate transformation $\phi_t$ transform the volume into\footnote{compare with Heisenberg's picture, what's the ``interaction picture''?(I don't know)}
\[
    \det(D\phi_t(\b x)\b h_1, D\phi_t(\b x)\b h_2,\cdots ,D\phi_t(\b x)\b h_{2d})=\Omega(t,\phi_t(\b x))
    \]
where the local reference frame sits on the point $\phi_t(\b x)$.

Under the same Hamiltonian, it is possible to change the notation ``$\Omega(t,\phi_t(\b x))$'' into ``$\Omega(t,\b x)$'' without causing confusion.
 When the linear operator $D\phi_t(\b x)$ is expressed in the matrix form with respect to the family $\mathfrak B$, the properties of the alternating tensors then gives
\begin{equation}\label{omega}
    \det(D\phi_t(\b x))\det(\b h_1, \b h_2,\cdots ,\b h_{2d})=\Omega(t, \b x)
\end{equation}
\def\trace{\operatorname{Trace}}
Notice that the determinant and trace of a matrix does not depend on the basis, since they are only associated with eigenvalues counting their multiplicities.\footnote{Or, $\det (AB)=\det (BA)$ and $\trace (AB)=\trace (BA)$}

If the matrix $D\phi_t(\b x)$ is expressed in the form of column vectors
\[
    D\phi_t(\b x)=[\b v_1, \b v_2, \cdots , \b v_{2d}]
\]
\def\Ah{\b J^{-1}\hex}
then there
\begin{align*}
&\td\relax\det(\b v_1, \b v_2, \cdots , \b v_{2d})\\
=&\det(\td{\b v_1}, \b v_2, \cdots , \b v_{2d})+\det(\b v_1,\td{\b v_2}, \cdots , \b v_{2d})+\cdots\\
=&\det(\Ah \b v_1, \b v_2, \cdots , \b v_{2d})+\det(\b v_1,\Ah \b v_2, \cdots , \b v_{2d})+\cdots
\end{align*}
At this point if $D\phi_t(\b x)$ is not singular\footnote{Assume that the degree of freedom unchanged}, then the matrix $D\phi_t(\b x)$ may be cast into the basis $\{\b v_1, \b v_2, \cdots , \b v_{2d}\}$
\begin{align*}
\det(a_{11}\b v_1+a_{12}\b v_2+\cdots, \b v_2, \cdots , \b v_{2d})+\det(\b v_1,a_{21}\b v_1+a_{22}\b v_2+\cdots, \cdots , \b v_{2d})+\cdots\\
\end{align*}
After doing some elementary operations, there\footnote{It is so lucky that by chance the proof of Liouville-Jacobi-Ostrogradskii identity is implied}
\begin{equation}\label{mu}
\td\relax\det(D\phi_t(\b x))=\trace(\Ah)\det(D\phi_t(\b x))
\end{equation}
Since the order of second derivatives can be changed, the trace would be zero. \mref{mu} gives
\begin{equation}\label{mucat}
\td\relax\det(D\phi_t(\b x))=0
\end{equation}
With $\det(D\phi_{t=0}(\b x))=\det (I_{2d})=1$, \mref{omega} gives
\begin{equation}\label{li1}
\Omega(0, \b x)=\Omega(t, \b x)
\end{equation}
It seems that Liouville's theorem lurks in the equation above. 

Define the $2d$ cube associated with $\mathfrak B$ as
\begin{equation}
\Gamma=\{\b z\in \mathbb R^{2d} :\b z=\b x(t)+\sum_{\b h\in \mathfrak B}\alpha_{\b h} \b h, \text{with }0\leq \alpha_{\b h}\leq 1\text{ for any }\b h \}
\end{equation}
It can be proved that the volume of the set $\Gamma$ is equal to $|\Omega(0,\b x)|$ using the tools from the measure theory\footnote{The proof is elementary except for the part involving the second countability, which could be perhaps understood intuitively}.

\def\vn{\mathcal N}
Assume there is an ensemble of systems labelled by $i=1,2,\cdots , \vn$\footnote{The letter ``$N$'' is reserved for other purposes}, each of them corrresponds to a ``generalized particles''. Define the density $\rho$ as
\begin{equation}
\rho=\frac{\text{number of systems in the cube }\Gamma}{|\Omega(0,\b x)|}
\end{equation}
$\rho$ is a function of time and $\b x$ and characterizes the density of the generalized particles.

With \mref{li1}, it appears that 
\begin{equation}\label{lio}
\td{\rho}=0
\end{equation}
The equation above is the Liouville's theorem. It may be differentiated to give
\begin{equation}
\left(\frac{\partial\rho}{\partial t}\right)_{\b q,\b p}+\sum_{k=1}^{d}\left(\frac{\partial\rho}{\partial q_k}\dot{q_k}+\frac{\partial\rho}{\partial p_k}\dot{p_k}\right)=0
\end{equation}
If a property $f$ relies on microscopic variable $\b x$, then under the assumption that each generalized particle is equally probable, the average of $f$, in close connection with the macroscopic properties, is calculated as
\begin{equation}\label{mac}
<f>=\frac{\int f\rho d\Omega}{\vn}
\end{equation}
The integration may span different dimensions, that is, taking the sum of the integrations of each entire spaces with different dimensions. 

The property $f$ is a function on $\mathbb R^{2d}$, and $\vn$ is unchanged with time. If $\rho$ changes with time while keeping $\b x$ unchanged, then the average of $f$ may change also. If the system is 
in the equilibrium, then $\rho$ is not likely to change with time
\[
    \left(\frac{\partial\rho}{\partial t}\right)_{\b q,\b p}=0
    \]
With Liouville's equation, it is thus meaingful to find the solution to the equation
\begin{equation}\label{sta}
\sum_{k=1}^{d}\left(\frac{\partial\rho}{\partial q_k}\dot{q_k}+\frac{\partial\rho}{\partial p_k}\dot{p_k}\right)=0
\end{equation}

The equation above may be written as
\[
    D\rho(\b x)\cdot\dot{\b x}=0
    \]
 Therefore if the generalized particle has nonzero velocity, then it is sufficient that at any instant the value of $\rho$ along the trajectory\footnote{or the ``flow line''} predicted by \mref{he} keeps unchanged.

Consider the real valued function $g$ on $\mathbb R^{2d}$ with the following properties

{\itshape Let $G$ be a real variable and $S=\{\b x : g(\b x)=G\}$. $\phi_t$ maps the set $S$ to itself all the time.

In other words, $\phi_t(g^{-1}(\{G\}))\subset g^{-1}(\{G\})$ for all $0\leq t$ }

It is possible that a list of such functions may be found
\[
    g_1, g_2,\cdots ,g_n
    \]
Because
\[
    \phi_t(\bigcap_{k=1}^{n}g_n^{-1}(\{G_n\}))\subset \bigcap_{k=1}^{n}\phi_t(g_n^{-1}(\{G_n\}))\subset\bigcap_{k=1}^{n}g_n^{-1}(\{G_n\})
    \]
If define\[
    [G_1,G_2,\cdots ,G_n]=\bigcap_{k=1}^{n}g_n^{-1}(\{G_n\})
    \]
    then the trajectory of the general particles cannot escape from the set $[G_1,\cdots ,G_n]$.

One possible solution to the equation \mref{sta} is to let $\rho$ be a function of $G_1,G_2,\cdot ,G_n$ only
\begin{equation}\label{sol}
\rho=\rho(G_1,G_2,\cdots ,G_n)
\end{equation}
Once the generalized particle is ``trapped'' in the set $[G_1,G_2,\cdots ,G_n]$, it cannot escape. Therefore the trajectory lines are all within the set and along the trajectory $G_1,G_2,\cdots ,G_n$ keep unchanged.
Consequently, the solution given by \mref{sol} has constant density along the trajectory. Therefore, it appears that \mref{sol} is a solution for the equilibrium state.

Examples of the functions that share the property of $g$ includes the energy, which comes from energy conservation; volume of the system, which assigns for each point $\b x$ the volume in the usual sense; dimension and number of particles of the systems, which is constrained by the algebraic expression of the Hamiltonian when it is time independent\footnote{$H$ only involves $\b x$ and has no time argument }.

The solution given by \mref{sol} may be written as
\begin{equation}\label{sol2}
\rho=\rho(g_1(\b x),g_2(\b x),\cdots ,g_n(\b x))
\end{equation}
without causing confusion.

Products of the solutions of the form as in \mref{sol2} is a solution, so are sum, composition of functions, etc. 

In general, the solution takes the following explicit form
\begin{equation}\label{sol3}
\rho=\int c(G_1,G_2,\cdots ,G_n)\cdot 1_{[G_1,G_2,\cdots ,G_n]}dG_1dG_2\cdots dG_n
\end{equation}
The integration may span different dimensions. $1_{[G_1,G_2,\cdots ,G_n]}$ is the elementary step function which assigns to each ``compartment'' $[G_1,G_2,\cdots ,G_n]$ value $1$ and zero elsewhere.
The coefficient $c(G_1,G_2,\cdots ,G_n)$ may involve Dirac's $\delta$ function\footnote{Not only in the case of the functions $g$ having discrete values such as dimensions and number of particles of the systems}.

It is desirable to further investigate the mathematical expression of the coefficient $c(G_1,G_2,\cdots ,G_n)$. To this end, a g-space $\mathbb R^n$ is set up. 
The space may be ``discretized'' into small cube and one of them is
\begin{equation}
    G_{i_1,i_2,\cdots ,i_n}=\prod_{k=1}^{n}[G_{i_k},G_{i_k}+\epsilon)
    \label{cubi}
\end{equation}
$(g_1,g_2,\cdots ,g_n)^{-1}(G_{i_1,i_2,\cdots ,i_n})$ may be empty set if some of the half-open interval are disjoint from the image of the domain of some $g$ function such as dimensions.

Because the set of the families $\{i_1,i_2,\cdots ,i_n\}$ is countable, it is without loss of generality to replace this notation with $i$ only.

Let $N_i$ denote the number of the generalized particles that are trapped in the set
\begin{equation}
    N_i=\text{card}((g_1,g_2,\cdots ,g_n)^{-1}(G_i))
    \label{number}
\end{equation}
when the $G_i$ cube is empty, $N_i$ is set to 1.

The map $(g_1,g_2,\cdots ,g_n)(\b x)=(g_1(\b x),\cdots ,g_n(\b x))$.

Because the generalized particles by the ensemble theory is independent and indistinguishable from each other as incomplete information is taken into consideration,
the probability that the certain distribution of the generalized particles is proportional to
\begin{equation}
   P\propto \frac{\mathcal N!}{\prod_{i}N_i!}
    \label{prob}
\end{equation}

It is therefore interesting to maximize $P$, or equivalently, for sufficiently large $\vn$, to extremize\footnote{Using $\ln (N!)\approx N\ln N-N$, for the proof, take the integral $\int_{1/N}^{1}\ln xdx$}%
\footnote{Besides the natural logarithm function, other strictly monotonic increasing function will give the same result. Since this fact can be understood intuitively, the proof is omitted here.}
\begin{equation}
    s=\sum_{i}N_i\ln N_i
    \label{gs}
\end{equation}
Consider the restraints prescribed by
\begin{equation}
    \sum_i N_iG_{i_{l}}=\overline{G_l}\text{\ \  , for } l=1,2,\cdots ,n
    \label{cons}
\end{equation}
where $\overline{G_l}$ are constants. With the help of Lagrange's multipliers, there\footnote{that's the reason why the formulations of canonical ensembles are similar to the Maxwell-Boltzmann's distribution}
\begin{equation}
    N_i=Ce^{\sum_l\alpha_lG_{i_{l}}}
    \label{lag}
\end{equation}
It may be restored to the continuous cases and obtain the solution of the form below\footnote{The central limit theorem is implied here.}
\begin{equation}\label{solution}
\rho=C\int e^{\alpha_1G_1+\alpha_2G_2+\cdots +\alpha_nG_n}\cdot\delta(G'_1-G_1)\delta(G'_2-G_2)\cdots\delta(G'_n-G_n)\cdot 1_{[G'_1,G'_2,\cdots ,G'_n]}dG'_1dG'_2\cdots dG'_n
\end{equation}
It may be criticizied that the formula in the form given by \mref{solution} is abstract enough to be useless for practical purposes after complicated mathematical reasoning.
To counter this argument, the solution \mref{solution} will be applied to the case of grand canonical ensemble to obtain the probability density function

The grand canonical ensemble concerns the systems with varying energy and varying number of particles of the systems, but keep the volume unchanged. The formula \mref{solution} reduced to
\begin{equation}
    \rho=C\int\int e^{\alpha_1E+\alpha_2N}\delta (E-E')\delta (N-N')\cdot 1_{[E',N']}dE'dN'
    \label{gce}
\end{equation}
It could be integrated by each variable separately
\begin{align}
    \rho&=C\int\int e^{\alpha_1E+\alpha_2N}\delta (E-E')\delta (N-N')\cdot 1_{[E',N']}dE'dN'\nonumber\\
    &=C\int e^{\alpha_1E}\delta (E-E')dE'\int e^{\alpha_2N}\delta (N-N')\cdot 1_{[E',N']}dN'\label{lastline}
\end{align}
In the last step \mref{lastline}, it is necessary to compute
\begin{equation}
    \int e^{\alpha_2N}\delta (N-N')\cdot 1_{[E',N']}dN'
    \label{ngce}
\end{equation}
If $N$ is not a natural number, then the set $[E',N']$ would be empty because the dimensions cannot assign a non-integer to a generalized particle.
Therefore there is no set for assigning number 1, the elementary step function equals to zero, the $\rho=0$. If $N$ is a natural number, then the Dirac's $\delta$ function gives
\begin{align}
    \rho &=C\int e^{\alpha_1E+\alpha_2N}\delta (E-E')\cdot 1_{[E',N]}dE'\nonumber\\
    &=Ce^{\alpha_1E+\alpha_2N}
    \label{calgce}
\end{align}
The elementary step function is 1 because at any point $\b x$ the energy of the system can be calculated.

The average of $f$ then has the form
\begin{equation}
    <f>=\int f\rho d\Omega =C\sum_{N=1}^{+\infty}\int f\cdot e^{\alpha_1E+\alpha_2N} d\Omega
    \label{gceaverage}
\end{equation}
The summation appears because when the $N$ is not an integer, the elementary step function equals to zero and the integral vanished between.
The theory developed by this note is then merged into the ``canonical'' discussion of the theory of statistical mechanics.

At this point one can see the profound implication of Liouville's theorem, and the significant of the Liouville's theorem is that when combined with
simple assumptions, the theorem gives many useful conclusions. Liouville's theorem is the first step to investigate the probrability distributions from the microscopic point of view to the macroscopic situations. 
